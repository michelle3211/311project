\documentclass[]{article}
\usepackage[T1]{fontenc}
\usepackage[margin=1in]{geometry}
\usepackage{lmodern}
\usepackage{amssymb,amsmath}
\usepackage{ifxetex,ifluatex}
\usepackage{fixltx2e} % provides \textsubscript
% use upquote if available, for straight quotes in verbatim environments
\IfFileExists{upquote.sty}{\usepackage{upquote}}{}
\ifnum 0\ifxetex 1\fi\ifluatex 1\fi=0 % if pdftex
  \usepackage[utf8]{inputenc}
\else % if luatex or xelatex
  \ifxetex
    \usepackage{mathspec}
    \usepackage{xltxtra,xunicode}
  \else
    \usepackage{fontspec}
  \fi
  \defaultfontfeatures{Mapping=tex-text,Scale=MatchLowercase}
  \newcommand{\euro}{€}
\fi
% use microtype if available
\IfFileExists{microtype.sty}{\usepackage{microtype}}{}
\ifxetex
  \usepackage[setpagesize=false, % page size defined by xetex
              unicode=false, % unicode breaks when used with xetex
              xetex]{hyperref}
\else
  \usepackage[unicode=true]{hyperref}
\fi
\hypersetup{breaklinks=true,
            bookmarks=true,
            pdfauthor={CPSC 311 Team T3},
            pdftitle={Background Report: TypeScript Webapp -- Linkbait Article Generator (Resubmission)},
            colorlinks=true,
            citecolor=blue,
            urlcolor=blue,
            linkcolor=magenta,
            pdfborder={0 0 0}}
\urlstyle{same}  % don't use monospace font for urls
\setlength{\parindent}{0pt}
\setlength{\parskip}{6pt plus 2pt minus 1pt}
\setlength{\emergencystretch}{3em}  % prevent overfull lines
\setcounter{secnumdepth}{0}

\title{Background Report: TypeScript Webapp -- Linkbait Article Generator (Resubmission)}
\author{CPSC 311 Team T3}
\date{November 27, 2015}

\begin{document}
\maketitle

Name: Ashley Lee\\Ugrad ID: k7y8\\Student ID: 34959122\\Email:
alee238@hotmail.com

Name: Min Seok Ray Roh\\Ugrad ID: l9x9a\\Student ID: 33737123\\Email:
msroh94@gmail.com

Name: Michelle Wilson\\Ugrad ID: a2d0b\\Student ID: 60855087\\Email:
m888wilson@gmail.com

Name: Cecile Leung\\Ugrad ID: o8r6\\Student ID: 90600974\\Email:
cecilephl@gmail.com

Name: Norman Sue\\Ugrad ID: h0e9\\Student ID: 20396131\\Email
normansue3@gmail.com

\section{Project Overview}\label{project-overview}

For the 100\%-level milestone, we would complete the following:

\begin{enumerate}
\def\labelenumi{\arabic{enumi}.}
\itemsep1pt\parskip0pt\parsep0pt
\item
  Write and publish an \texttt{npm} module in TypeScript which generates
  a JSON representation of linkbait articles by taking advantage of type
  checking.
\item
  Write missing TypeScript type definition files for third-party
  \texttt{npm} modules that our \texttt{npm} module depends on.
\item
  Verify that all of our written TypeScript type definition files are
  bugfree with
  \href{https://github.com/asgerf/tscheck}{tscheck}.\cite{ref1}
\item
  Use our \texttt{npm} module to create a simple Node.js+Express API
  server that runs the web scraper module and sends the resulting JSON
  articles to the browser client.
\item
  Write a TypeScript single-page application browser client that renders
  the calculated backend JSON API data with jQuery and Backbone.js.
\end{enumerate}

An explanation of how what each of these components are and how they
showcase specific TypeScript language features is given in the following
subsections.

\subsection{Write and publish an \texttt{npm} module in TypeScript which
generates a JSON representation of linkbait articles by taking advantage
of type
checking.}\label{write-and-publish-an-npm-module-in-typescript-which-generates-a-json-representation-of-linkbait-articles-by-taking-advantage-of-type-checking.}

We've chosen to publish our project as an \texttt{npm} module because,
according to \href{http://www.modulecounts.com/}{modulecounts.com}
\cite{ref2}, it's the industry-standard JavaScript (and
thus TypeScript) package management format for running code on
\href{https://nodejs.org/en/}{Node.js}, the JavaScript runtime
environment that allows our compiled TypeScript code to be executed
server-side. By using a widely-accepted, industry-standard library
format, we can ensure the core functionality of our project can be
reused by other programmers.

Specific TypeScript language features to be used by the module include:

\subsubsection{Functions}\label{functions}

As TypeScript is a superscript of JavaScript, there are similarities
between them, however, when writing the former, there are differences
that will compile into the latter. These differences may give
flexibility in writing functions. Undoubtedly, when coding our linkbait
article generator, we're going to end up using functions to represent
the behavior of the generator, while probably encapsulating those in
classes. TypeScript has added functionality to functions that JavaScript
does not, such as binding the context of `this', contextual typing, and
the ability for optional and default parameters.
\cite{ref3}  These capabilities make it easier to code and
maintain a typed program, which will help us maintain our code better
than if we coded in JavaScript.

\subsubsection{Generics}\label{generics}

In TypeScript, if a generic function is created, the compiler will
enforce that all the actions taken in the function are used in such a
way that they \emph{could} work with all types. So if you attempt an
operation that is only allowed for type \texttt{String} but not
\texttt{Object}, it will be forbidden (causes an
error).\cite{ref3} But this is recognized only at compile
time. There is no run-time representation for type
parameters.\cite{ref3bi}

Despite this, static-checking of the appropriate usage of generic types
will be useful for when we iterate over possible articles in a
\texttt{for... each} loop, to perform bulk formatting such as inserting
abbreviations for our headlines.

\subsubsection{Mixins}\label{mixins}

Mixins are a way of reusing code to make new classes by combining
desired parts of existing classes without taking on all features of
those classes. The specificity of which methods are retained avoids the
problems from ambiguity that comes up in multiple inheritance otherwise.
They are useful in situations where a particular behaviour is repeated
in many classes, providing optional behaviour in a class, and making
variations on similar features in the augmented class.
\cite{ref3}

\subsubsection{Intersection Types}\label{intersection-types}

By using the \texttt{pjscrape} web scraping library, we will collect an
\texttt{Array} of parsed data from other popular articles, which will
ensure that our generated articles' headlines will have a high chance of
being clicked on.

However, the types of the objects within the collected \texttt{Array}
data will be unknown ahead of time, aside from the fact that they could
be a mathematical union of a discrete number of types. TypeScript's
intersection types feature \cite{ref4} allows us to
statically check that our subsequent functions use the parsed
\texttt{Array} elements in a way that's consistent with JavaScript's
built-in methods on native library types. For example, we can only call
\texttt{String.prototype.concat} on \texttt{String} types, but the
elements may be both \texttt{String} and \texttt{Number} types, since
some of the articles we parse may contain numeric data.

If we were to write our subsequent data cleaning code without the usage
of TypeScript's static type checking features, it may result in
JavaScript \texttt{TypeError}s.

\subsection{Write missing TypeScript type definition files for
third-party \texttt{npm} modules that our \texttt{npm} module depends
on.}\label{write-missing-typescript-type-definition-files-for-third-party-npm-modules-that-our-npm-module-depends-on.}

We plan on scraping the text content of existing websites in order to
generate our own linkbait articles. This will require using the existing
\href{https://github.com/nrabinowitz/pjscrape}{pjscrape}
\cite{ref5} JavaScript web scraping library.
\texttt{pjscrape} currently does not have TypeScript type definitions
within the
\href{https://github.com/DefinitelyTyped/DefinitelyTyped}{DefinitelyTyped}
\cite{ref6} repository, so we'll need to inspect their
\texttt{export}ed functions and write \texttt{module} and
\texttt{interface} definitions for each of them.

\subsection{Verify that all of our written TypeScript type definition
files are bugfree with
\href{https://github.com/asgerf/tscheck}{tscheck}.}\label{verify-that-all-of-our-written-typescript-type-definition-files-are-bugfree-with-tscheck.}

\texttt{tscheck} is an existing JavaScript library based on research by
Feldthaus and Møller that can be used to find bugs in handwritten
TypeScript type definition files. Running this check ensures that the
\texttt{module} and \texttt{interface} definitions that we need to write
for the required \texttt{npm} modules (e.g. \texttt{pjscrape}) are
bugfree, such that when our code calls their functions, TypeScript's
\texttt{tsc} compiler will correctly perform static type checking.
\cite{ref1}, \cite{ref7}

\subsection{Use our \texttt{npm} module to create a simple Node.js +
Express API server that runs the web scraper module and sends the
resulting JSON articles to the browser
client.}\label{use-our-npm-module-to-create-a-simple-node.js-express-api-server-that-runs-the-web-scraper-module-and-sends-the-resulting-json-articles-to-the-browser-client.}

\href{http://expressjs.com/}{Express.js} is the most commonly used
Node.js library for creating minimalistic API servers. We will use it to
serve the HTML, CSS and compiled JavaScript browser client.

\subsection{Write a TypeScript single-page application browser client
that renders the calculated backend JSON API data with jQuery and
Backbone.js.}\label{write-a-typescript-single-page-application-browser-client-that-renders-the-calculated-backend-json-api-data-with-jquery-and-backbone.js.}

Our browser client will use \href{http://backbonejs.org/}{Backbone.js},
one of the popular client-side JavaScript frameworks for providing
structure to the registration of jQuery callback functions to the
various DOM elements that we'll use to render our API's JSON data
containing the generated linkbait articles.

\section{Project Value}\label{project-value}

Of the most popular compile-to-JS languages, TypeScript greatly lags
behind in adoption and has significantly fewer exemplary repositories to
draw inspiration from. There are currently 8,267 TypeScript repositories
on GitHub, compared with the 1.7 million JavaScript and 54,605
CoffeeScript (the leading compile-to-JS language that competes with
TypeScript) repositories.

Creating an additional TypeScript project provides future potential
TypeScript programmers with an additional example of how they could
utilize TypeScript's static typing features.

\section{Project Importance}\label{project-importance}

\subsection{Background}\label{background}

Clickbait is a term for headlines that catch people's attention and
curiosity enough for them to follow a link. The main goal is for the
user to click through to the target website. Secondary is the content,
which has a reputation for being of very low quality. They exist due to
a business model whereby the more visitors there are to a website, the
more advertisers are willing to pay to use that site. Ads on that site
gain more visibility with each visitor and can have further referral
links. More visitors can also affect a site's rank when it shows up on
search results due to search engine optimization (SEO) rules. Therefore,
any click-through on a link can have compounding effects on visibility
and subsequently revenue from advertising.

\subsection{Negatives and possible positive
side-effects}\label{negatives-and-possible-positive-side-effects}

Clickbait (also known as linkbait) headlines may read like news, but the
articles are often hastily created with little research, no insight,
misleading information, or they can be outright advertisements
masquerading as impartial articles. It can annoy consumers
\cite{ref8} and it can also affect the way content is
created, encouraging poor journalism \cite{ref9}. So, there
is little \emph{inherent} value to creating clickbait. However, we may
gain some beneficial knowledge as a byproduct. Creating successful
clickbait involves the psychology of curiosity and persuasion. If we can
learn more about what works, we can put it to good use (without annoying
the readership) by providing high quality content at the target site.
For example, actual information.

\subsection{Our perspective}\label{our-perspective}

We chose this as our application because it would be an amusing way to
showcase the features of TypeScript. However, it could be extended to
track data on which generated headlines are successful (number of click
throughs and duration spent on target sites). This information could be
used to learn how to best reach certain audiences with information. It
could be anything from public health outreach to how undergrads can
apply for co-op. Another possible benefit is that our team could sign up
for ad revenue and make money for this project.

\section{Project Impact}\label{project-impact}

Standardized support for writing TypeScript \texttt{npm} modules that
compile to JavaScript by using
\href{https://github.com/Microsoft/TypeScript/wiki/tsconfig.json}{tsconfig.json
files} to integrate with existing JavaScript and TypeScript \texttt{npm}
modules has only been added 4 months ago in TS 1.5.
\cite{ref4}

By publishing a library using this relatively-new build process, we are
contributing to the TypeScript community by providing an additional
working example of how to use this new build feature of the language,
since there are only 678 results \cite{ref10} when
searching for GitHub code that use tsconfig.json files.


\begin{thebibliography}{15}
\bibitem{ref1}
Feldthaus, Asger. ``asgerf/tscheck (Git code repo).'' 15 Aug
2014. 16 Nov 2015. \url{https://github.com/asgerf/tscheck}

\bibitem{ref2}
DeBill, Erik. ``Module Counts.'' 16 Nov 2015 (data updated
daily). \url{http://www.modulecounts.com/}

\bibitem{ref3}
``TypeScript Handbook.'' \emph{Microsoft}. 9 Nov. 2015. 12 Nov.2015.
\url{https://github.com/Microsoft/TypeScript-Handbook/blob/master/pages/Functions.md\#lambdas-and-using-this}, \url{https://github.com/Microsoft/TypeScript-Handbook/blob/master/pages/Generics.md}, \url{https://github.com/Microsoft/TypeScript-Handbook/blob/master/pages/Mixins.md}

\bibitem{ref3bi}
``TypeScript: Language Specification.'' \emph{Microsoft}. Feb
2015. 12 Nov. 2015.
\url{http://www.typescriptlang.org/Content/TypeScript\%20Language\%20Specification.pdf}

\bibitem{ref4}
``What's new in TypesScript.'' \emph{Microsoft}. 13 Nov 2015. 16Nov 2015.
\url{https://github.com/Microsoft/TypeScript/wiki/What/'s-new-in-TypeScript\#intersection-types}, \url{https://github.com/Microsoft/TypeScript/wiki/What/'s-new-in-TypeScript\#typescript-15}

\bibitem{ref5}
Rabinowitz, Nick. ``nrabinowitz/pjscrape (Git code repo).'' 23
May 2014. 16 Nov 2015. \url{https://github.com/nrabinowitz/pjscrape}

\bibitem{ref6}
``DefinitelyTyped: The repository for high quality TypeScript
type definitions.'' 23 May 2014. 16 Nov 2015.
\url{https://github.com/DefinitelyTyped/DefinitelyTyped} also
\url{http://definitelytyped.org/}

\bibitem{ref7}
Feldthaus, Asger. ``Checking Correctness of TypeScript
Interfaces for JavaScript Libraries''
\url{https://cs.au.dk/~amoeller/papers/tscheck/paper.pdf}

\bibitem{ref8}
Frampton, Ben. ``Clickbait: The changing face of online
journalism.'' \emph{BBC News}. 14 Sep 2015. 16 Nov 2015.
\url{http://www.bbc.com/news/uk-wales-34213693}

\bibitem{ref9} Shire, Emily. ``Saving Us From Ourselves: The Anti-Clickbait
Movement.'' \emph{Daily Beast}. 14 Jul 2014. 16 Nov 2015.
\url{http://www.thedailybeast.com/articles/2014/07/14/saving-us-from-ourselves-the-anti-clickbait-movement.html}

\bibitem{ref10}
GitHub search performed 16 Nov 2015.
\url{https://github.com/search?l=typescript\&q=tsconfig\&type=Code\&utf8=\%E2\%9C\%93}
\end{thebibliography}
\end{document}
